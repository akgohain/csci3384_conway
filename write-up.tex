\documentclass{article}
\usepackage[utf8]{inputenc}
\usepackage[margin=1in]{geometry}
\usepackage{amsmath}
\usepackage{amsthm}
% Package for making turing machine diagrams %
\usepackage{tikz}
\usetikzlibrary{chains,fit,shapes}
% Packages for algorithms %
\usepackage{algorithm}
\usepackage{algorithmic}
% Package which has the nice looking empty set symbol (\varnothing)
\usepackage{amssymb}
% Package with the ceiling function
%\usepackage{mathtools}
%\DeclarePairedDelimiter{\ceil}{\lceil}{\rceil}
\usepackage{braket}
\usepackage{amsmath} 
\usepackage{amsfonts}
\usepackage{amssymb}
\usepackage{comment}
\usepackage{mathtools}
\DeclarePairedDelimiter{\ceil}{\lceil}{\rceil}
\usepackage{bm}

\usepackage{biblatex}
\addbibresource{bibliography.bib}

% Makes table of contents links clickable in pdf readers
\usepackage[colorlinks=true, linkcolor=blue, urlcolor=blue, citecolor=blue, breaklinks=true]{hyperref}

\theoremstyle{definition}
\newtheorem{definition}{Definition}[section]
\newtheorem{problem}{Problem}

\theoremstyle{plain}
\newtheorem{example}{Example}[section]
\newtheorem{exercise}{Exercise}[section]

\theoremstyle{plain}
\newtheorem{fact}{Fact}[section]
\newtheorem{lemma}{Lemma}[section]
\newtheorem{theorem}{Theorem}[section]
\newtheorem{corollary}{Corollary}[section]
\newtheorem{claim}{Claim}[section]

\title{Reachability, Complexity, and the Limits of Conway's Game of Life}
\author{Sebastian Pucher \& Adam Gohain}

\begin{document}

\maketitle

\tableofcontents

\newpage

\section{Introduction: Can Life be Simulated?}
  \textit{    That is the question, isn't it.} For centuries, mathematicians, philosophers, artists, and computer scientists, have spent their lives trying to uncover what it means to truly be alive. Many come together, often crossing their respective disciplines to construct answers to these larger, often existential questions. Even today, humanity has continued to develop complicated technology in hopes of understanding more about life, how it can be studied, or even how it could possibly be synthesized.

\

\textit{What is life?} Back in 1940, John Von Neumann and Stanislaw Ulam set out to prove an answer to this very question. Von Neumann was an American Mathematician whose research focused on self-replicating systems and cellular automata. Alongside his colleague Ulam, who worked together with Von Neumann on the Manhattan Project, they proposed a simple discrete game that replicated life. Their mathematical model consisted of a two-dimensional grid of square cells, where the state of the next generation of cells would depend on the interaction between living cells and their neighbors \cite{Beginning_Life_2006}. They called it \textit{The Universal Constructor} which produced fascinating properties of time and space usage \cite{Freitas_2004}.

\

Not long after their proposal, a British Mathematician known as John Conway extended upon Ulam and Von Neumann’s research to fabricate an instance of their Universal Constructor that better replicated Alan Turing’s “universal computer”. By experimenting with different rules and states between neighboring automata, Conway was able to simplify the model into a game that was only composed of only a few basic rules \cite{Beginning_Life_2006}. [see section (CITE SECTION)]. 

\

Shortly after, in 1970, the \textit{Scientific American} published an article articulating how to play the game which resulted in the greatest number of letters reactions from readers at that time \cite{Izhikevich_Conway_Seth}. In the paper, Conway proposed that no initial pattern could grow without limit, and offered fifty dollars to the first person who could disprove him by the end of the year \cite{math-games}. This catalyzed immense popularity in the game, and set forth the many mathematical discoveries that have now been proven about the game, such as its undecidable nature (see section (CITE SECTION)), and recurring patterns(see section).


\textbf{Going to add a nice image here}

\subsection{The Game}
Similar to how Alan Turing proposed models of computational thinking prior to modern day computers, Conway's Game of Life started as a mathematical idea that was “played” on chalkboards and Go boards \cite{Izhikevich_Conway_Seth}. Here are the rules, and how to play: 

\

\textbf{Rules \& Properties: }
\begin{enumerate}

  \item \textbf{Domain: }\\ The automata in the game interact within an \textit{infinite} two-dimensional grid of cells. Every cell has eight neighbors \cite{Izhikevich_Conway_Seth}[figure x].

  \item \textbf{States: }\\ Each automata is represented independently by a single cell which can be either \textit{alive} or \textit{dead}.

  \item \textbf{Initial Configuration: } \\ The beginning set of live or dead cells is determined or "seeded" by the player prior to any evolution.

  \item \textbf{Evolution: } \\ The following rules are applied to all cells simultaneously in fixed time intervals called \textit{generations} \cite{Bontes2019}.

  \item \textbf{Birth: } \\ A new cell is born at generation $t + 1$ when its state is currently \textit{dead} and has exactly three lives neighbors (reproduction) \cite{Bontes2019}.

  \item \textbf{Death: } \\ Any currently living cell will die at generation $t + 1$ if it has less than 2 live neighbors (underpopulation) or more than three live neighbors (overpopulation) \cite{Bontes2019}.

  \item \textbf{Persistence: } \\ Any live cell will persist at $t + 1$ if it has two or three live neighbors at generation $t$ \cite{Izhikevich_Conway_Seth}.
\end{enumerate}

It’s important to note that Conway’s Game of Life is not a game of how we traditionally think of games. There’s no objective, or winning or losing. They’re aren’t even any players – it is known to be a zero player game \cite{Beginning_Life_2006}. As technology advanced, Conway’s Game of Life proved to be well-suited for implementation on computers. Today, the game has been optimized to explore the many unresolved problems and complexity the game poses. 

\subsection{So... What's the Big Deal?}
\section{Definitions, Theorems, and other Important Terminology}
\begin{enumerate}
  \item[] \textbf{GOALS FOR SUB SECT}
  \item Explain the common patters / life forms that can be found in the game
  \item Explain the three patterns we will be examining (Still life, ociliator, and spaceship)
  \item Add images to these 
  \item Explain other relevant mathematical terminology / information that's worth introducing 

\end{enumerate}

\subsection{What about Complexity?}
\begin{enumerate}
  \item[] \textbf{GOALS FOR SUB SECT}
  \item More of an intro sub section connecting the game to complexity analysis  
  \item This is where we'll define important compelexity classes like NP
  \item outline any other important info we want to outline before getting into more gritty / mathy stuff

\end{enumerate}

\section{Reaches-Configuration}

\subsection{Still life Analysis}
\begin{enumerate}
  \item[] \textbf{GOALS FOR SECTION}
\end{enumerate}

\subsection{Oscillator Analysis}
\begin{enumerate}
  \item[] \textbf{GOALS FOR SECTION}
\end{enumerate}

\subsection{Spaceship Analysis}
\begin{enumerate}
  \item[] \textbf{GOALS FOR SECTION}
\end{enumerate}

\section{What can be computed/decided in Conway’s Game of Life?}

\section{REACHES-CONFIGURATION is undecidable}

\printbibliography

\end{document}
