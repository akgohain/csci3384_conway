\documentclass{article}
\usepackage[utf8]{inputenc}
\usepackage[margin=1in]{geometry}
\usepackage{amsmath}
\usepackage{amsthm}
% Package for making turing machine diagrams %
\usepackage{tikz}
\usetikzlibrary{chains,fit,shapes}
% Packages for algorithms %
\usepackage{algorithm}
\usepackage{algorithmic}
% Package which has the nice looking empty set symbol (\varnothing)
\usepackage{amssymb}
% Package with the ceiling function
%\usepackage{mathtools}
%\DeclarePairedDelimiter{\ceil}{\lceil}{\rceil}
\usepackage{braket}
\usepackage{amsmath} 
\usepackage{amsfonts}
\usepackage{amssymb}
\usepackage{comment}
\usepackage{mathtools}
\DeclarePairedDelimiter{\ceil}{\lceil}{\rceil}
\usepackage{bm}

\usepackage{biblatex}
\addbibresource{bibliography.bib}

% Makes table of contents links clickable in pdf readers
\usepackage[colorlinks=true, linkcolor=blue, urlcolor=blue, citecolor=blue]{hyperref}

\theoremstyle{definition}
\newtheorem{definition}{Definition}[section]
\newtheorem{problem}{Problem}

\theoremstyle{plain}
\newtheorem{example}{Example}[section]
\newtheorem{exercise}{Exercise}[section]

\theoremstyle{plain}
\newtheorem{fact}{Fact}[section]
\newtheorem{lemma}{Lemma}[section]
\newtheorem{theorem}{Theorem}[section]
\newtheorem{corollary}{Corollary}[section]
\newtheorem{claim}{Claim}[section]

\title{Reachability, Complexity, and the Limits of Conway's Game of Life}
\author{Sebastian Pucher \& Adam Gohain}

\begin{document}

\maketitle

\tableofcontents

\newpage

\section{Abstract}
We'll probably do this last!
\section{Introduction: Can Life be Simulated?}

That is the question isn't it. For centries, mathemateions, 
and even soon computer scientis cross their respective dispipinaries in hopes of building an answer to this question. Even today, humanity has contintues to construct complicated technoilgy that hopes to answer these larger question. 

\textit{What is life?} Back in 1940, Ulam and Von Neumann set out to prove of an answer to this question by constructing model of artifical life. At the time, Ulam was conducting research into  



And even today, 
Through both empirical examples, as well as theoretical support th
\subsection{The Game}
Simularly to how Alan Turing proposed models of computation prior to the construction of what we know today as a computer, the specfic rules and methods of playing of Conway's Game of Life, predated any deep techonolgocal expcation of the future. 

Very similarly to how models of computation existed before computers did, so did ways of examining theories of life. The game of life was 
 
\begin{enumerate}
  \item[] \textbf{GOALS FOR SUB SECT}
  \par Conway's Game of Life (GoL for short)... 
  \item Outline basics of the game, how it works, a nice lil intro 
  \item Explain brief about the games significance in computer science (connect to theory)
  \item Intro to the games complexity characteristics 

\end{enumerate}
\subsection{Definitions, Theorems, and other Important Terminology}
\begin{enumerate}
  \item[] \textbf{GOALS FOR SUB SECT}
  \item Explain the common patters / life forms that can be found in the game
  \item Explain the three patterns we will be examining (Still life, ociliator, and spaceship)
  \item Add images to these 
  \item Explain other relevant mathematical terminology / information that's worth introducing 

\end{enumerate}

\subsection{What about Complexity?}
\begin{enumerate}
  \item[] \textbf{GOALS FOR SUB SECT}
  \item More of an intro sub section connecting the game to complexity analysis  
  \item This is where we'll define important compelexity classes like NP
  \item outline any other important info we want to outline before getting into more gritty / mathy stuff

\end{enumerate}

\section{Reaches-Configuration}

\subsection{Still life Analysis}
\begin{enumerate}
  \item[] \textbf{GOALS FOR SECTION}
\end{enumerate}

\subsection{Oscillator Analysis}
\begin{enumerate}
  \item[] \textbf{GOALS FOR SECTION}
\end{enumerate}

\subsection{Spaceship Analysis}
\begin{enumerate}
  \item[] \textbf{GOALS FOR SECTION}
\end{enumerate}

\section{What can be computed/decided in Conway’s Game of Life?}

\section{REACHES-CONFIGURATION is undecidable}

 Keep as reference for now -- Can remove later
\section{Basics of Structuring a LaTeX Paper}

\subsection{Definitions and Theorems}
You can organize your write-up into both sections and subsections. There are also subsubsections, if you really want them. 

If we want to present a theorem, lemma, corollary, etcetera, we enter into the appropriate environment, and then follow it up with the proof via a proof environment, like so:

\begin{theorem}
    $\mathbf{P} = \mathbf{NP}$
\end{theorem}

\begin{proof}
    It came to me in a dream.
\end{proof}

As you can see, theorems, lemmas and so forth are automatically numbered based on your sections and subsections. 

I personally like to present definitions within a theorem environment, for easy referencing.
\begin{definition}\label{NC}
    $\mathbf{NC}$ (standing for \emph{not cool}) is the set of decision problems that I personally dislike. 
\end{definition}

By using the \emph{label} command, you can reference anything you want later using the \emph{ref} command, e.g. if we consult definition \ref{NC} we can see...

\subsection{Citations and Bibliography}

Citations are stored in the bibliography.bib file. To show you an example, I've added the textbook linked on canvas. You can cite an entry in the bibliography using the cite command. E.g. according to chapter 7 of \cite{papadimitriouComputationalComplexity1994}...

There are plenty of online tools for automatically generating bibtex entries. Usually these can be directly copied from the publisher page for a paper. The references will be automatically generated when you compile. Be aware that \emph{references will only be added for the entries of your bibliography that you actually cite}. If I did not cite the textbook in this file, it would not appear as a reference. Don't let this confuse you!
\printbibliography

\end{document}
